%%%%%Präambel%%%%%

\documentclass[12pt,a4paper]{report}%Schriftgröße, Papierformat einstellen
%\documentclass{scrbook}
\usepackage[top=30mm,bottom=30mm]{geometry}
\usepackage{lipsum}
\usepackage{csquotes}
%Pakete laden zur deutschen Rechtschreibung und für Umlaute
\usepackage[T1]{fontenc}
\usepackage[ngerman]{babel}
\usepackage[utf8]{inputenc} %für Windows, Linux
%\usepackage[applemac]{inputenc} %für Mac
%\usepackage{xcolor}
\usepackage[official]{eurosym}
\usepackage{graphicx}
\usepackage{caption}
\usepackage[dvipsnames]{xcolor}
\usepackage{cancel}
\usepackage{titlesec}
\usepackage{cite}
\usepackage{filecontents}
\usepackage{tabularx}
\usepackage{harvard}
\usepackage{units}
\usepackage{longtable} 
\usepackage{multirow}
\usepackage{chngcntr}
\usepackage{stmaryrd}
\usepackage{array}
\usepackage{autobreak}
\usepackage{booktabs}
\usepackage{float}
\usepackage{wrapfig}
\usepackage{hhline}
\let\harvardleftorig\harvardleft
%\usepackage[round]{natbib}
%\usepackage{hyperref}
\usepackage[nottoc,numbib]{tocbibind}
\usepackage[nottoc]{tocbibind}
\usepackage{siunitx}
\usepackage{esvect}
\usepackage{trfsigns}

%Pakete laden zu mathematischen Symbolen etc.
\usepackage{calc} 
\usepackage{amsmath,amssymb,amsthm,amsopn}
\usepackage{scrpage2}
\pagestyle{scrheadings}
\clearscrheadfoot
\automark[chapter]{section}
\ofoot{\pagemark}
\ifoot{}
\chead{\headmark}
\setfootsepline{1pt}
\setheadsepline{1pt}
%\setheadsepline[\textwidth+20pt]{0.5pt}

%Inhaltsverzeichnis mit Links erstellen
\usepackage[colorlinks,
pdfpagelabels,
pdfstartview = FitH,
bookmarksopen = true,
bookmarksnumbered = true,
linkcolor = black,
plainpages = false,
hypertexnames = false,
citecolor = black] {hyperref}

\setcounter{secnumdepth}{4}
\setcounter{tocdepth}{4}

\titleformat{\paragraph}
{\normalfont\normalsize\bfseries}{\theparagraph}{1em}{}
\titlespacing*{\paragraph}
{0pt}{3.25ex plus 1ex minus .2ex}{1.5ex plus .2ex}

\newcommand{\subsubsubsection}{\paragraph}

\input{./tex/helpers/helpers.tex}
\input{./tex/helpers/maths.tex}
\input{./tex/helpers/citation.tex}
\input{./tex/helpers/formula_functions.tex}
\newcommand{\tabitem}{~~\llap{\textbullet}~~}

\renewcommand{\theequation}{\arabic{section}.\arabic{subsection}
.\arabic{equation}}
%Setzt den equation-Zaehler nach jeder Seite zurueck
%\numberwithin{equation}{subsection}	
\numberwithin{equation}{section}
%\setlength\abovedisplayskip{0pt}


%jetzt beginnt das eigentliche Dokument
\begin{document}
\bibliographystyle{agsm}

\author{}
\title{\underline{HM3 Kurzzusammenfassung} \\ $\;$ \\ $\;$ \\ Florian Leuze}
\date{}
\maketitle % erzeugt den Kopf

$\;$ \newline
$\;$ \newline
$\;$ \newline
$\;$ \newline
$\;$ \newline
$\;$ \newline
$\;$ \newline
$\;$ \newline
$\;$ \newline
$\;$ \newline
$\;$ \newline
$\;$ \newline
\begin{table}[H]
  \centering
  \begin{tabular}{P{14cm}}
    \begin{LARGE}
		  \glqq Was wirklich zählt 
    \end{LARGE}\\
    \begin{LARGE}
		   ist Intuition.\grqq
    \end{LARGE}\\
    \begin{large}
      (Albert Einstein)
    \end{large}
  \end{tabular}
\end{table}
\newpage
\tableofcontents

\section*{Versionierung}
\begin{tabular}{|p{2cm}|p{1cm}|p{1.5cm}|p{12.5cm}|}\hline
Datum & Vers. & Kürzel & Änderung \\ \hline
30.08.2019 & 0.1 & FL & Erzeugung Dokument; Erzeugung Inhaltsverzeichnis; Erzeugung Versionierung; Erzeugung Literaturverzeichnis; 1.1-1.8;\\ \hline
01.09.2019 & 0.110 & FL & 1.9 - 1.10 \\ \hline
02.09.2019 & 0.111 & FL & 1.11 \\ \hline
06.09.2019 & 0.112 & FL & 2.1; 2.2; 2.3 \\ \hline
09.09.2019 & 0.2 & FL & 2.3: 2.4; 2.5; 2.6; 2.7 \\ \hline
09.09.2019 & 0.21 & FL & small correction about sectioning \\ \hline
09.09.2019 & 0.22 & FL & layout change \\ \hline
\end{tabular}
\listoffigures


%
%%first chapter damn im too lazy to think about some good notes to put here
\newpage
\section{Grundlagen der Informationstheorie} 
	\section{Notationen}
	\subsection{Charakteristische Funktion}
	Die charakteristische Funktion einer Menge $A \subset \R^n$ ist gegeben durch:
	\begin{equation}
		\textsl{x}_A(x) = 
		\begin{cases}
			1 \qquad, x \in A\\
			0 \qquad, x \in \R^n_{\backslash A}
		\end{cases}
	\end{equation}
	
	Dabei gilt
	\begin{equation}
		\textsl{x}_A \cdot \textsl{x}_B = \textsl{x}_{A \cap B}
	\end{equation}
	
	\subsection{Kartesisches Produkt}
	\begin{equation}
		A \times B = \lbrace (a,b)| a \in A, b \in B \rbrace
	\end{equation}
	
	\subsection{Infimum/Supremum}
	Das Infimum einer Menge ist die größte untere Schranke, das Supremum die kleinste obere Schranke. Schreibweise:
	\begin{align}
		\inf_{x\in A} A \\
		\sup_{x \in B} B
	\end{align}
	\newpage
	\subsection{Randpunkt und Rand}
	Ein Randpunkt $a$ von $A \subset \R^n$ ist ein Punkt $a \in \R^n$ mit
	\begin{align}
		B_\varepsilon (a) \cap A \neq 0\\
		B_\varepsilon (a) \cap A^c \neq 0
	\end{align}
	für alle $\varepsilon > 0$. Der Rand $\partial A$ von $A$ ist die Menge aller Randpunkte.
	  \begin{figure}[H] 
		  \centering
		  \includegraphics[width=0.5\textwidth]{./img/mass_randpunkte.png}
		  \caption{Randpunkte \protect\cite{HM3}}
		  \label{fig:randpunkte}
	  \end{figure}

\subsection{Nützliche Hilfssätze}
	\subsubsection{Reduktionsformel Sinus}
	\begin{equation}
		\int_a^b \sin^n(x) \dx =
\frac{n-1}{n}\int_a^b \sin^{n-2}(x)\dx \quad, a,b \in \frac{\pi}{2} \cdot \Z
	\end{equation}		  
	
		
	
\section{Modulation cosinus-förmiger Träger}
	\subsection{Bandbreite}
\formTab{Basisbandbandbreite für reelle $x(t)$}{B = f_{max}}
\formTab{Bandbreite für komplexe $x(t)$}{B = 2\cdot f_{max}}

\subsection{Einzelimpuls}
Der Einzelimpuls ist gegeben durch
\begin{equation}
	g(t) \cdot T = \lbrace 
		\begin{cases}
		1 & ,\;t = 0\\
		\frac{\sin \left(\frac{\pi t}{T}\right)}{\pi \frac{t}{T}} & ,\;sonst
	\end{cases} \nonumber
\end{equation}
Nach \eqref{eq:fourier_corr_einzelimpuls} ist die Fouriertransformierte
\begin{equation}
	G(f) = \lbrace 
	\begin{cases}
		1 & ,\; |f| \leq \frac{R_S}{2} = \frac{1}{2T}\\
		0 & ,\; sonst
	\end{cases} \nonumber
\end{equation}

\begin{figure}[H]
	\centering
	\subfloat[t-domain]{\includegraphics[height=3cm]{./img/mod_einzel_t.png}} ~
	\subfloat[f-domain]{\includegraphics[height=3cm]{./img/mod_einzel_f.png}}
	\caption{Einzelimpuls \protect\cite{NT2}}
\end{figure}

\subsection{Amplitudenmodulation (AM)}
\formTab{Träger}{c(t) = \cos(w_0 t + \varphi_0)}
\formTab{Signal im Passband}{u(t) = \alpha_A \cdot x(t) \cdot c(t) = \alpha_A \cdot x(t) \cdot \cos(w_0 t + \varphi_0) \formTnQQQ = \Re \lbrace \underbrace{x_A \cdot x(t)}_{\text{kompl. Hüllkurve}} \cdot e^{j \varphi_0} \cdot e^{j w_0 t} \rbrace \formTnQQQ = \Re\lbrace x(t) \cdot e^{j w_0 t}\rbrace \quad \text{(für }\varphi = 0 \text{ und  } \alpha_A = 1 \text{)}}
Mit \eqref{eq:fourier_corr_am_1} erhält man durch nutzung der eulerschen Formel
\begin{equation}
	U(w) = \alpha_A \frac{1}{2}\left[X(w+w_0) + X(w-w_0)\right]
\end{equation}

\formTab{Signal mit Offset}{u(t) = \left( x(t) + A_{off} \right) \cdot \cos(w_0 t) \formTnQ = \left( x(t) + A_{off} \right) \left[\frac{1}{2} \cdot e^{j w_0 t} + \frac{1}{2} \cdot e^{-j w_0 t} \right]}
Mit \eqref{eq:fourier_corr_am_off} folgt
\begin{equation}
	x(t) + A_{off} \quad \laplace \quad X(w) + 2\pi A_{off} \delta(w)
\end{equation}

\subsubsection{Frequenzmultiplex}
\formTab{Fourier Identität}{u(t) = \sum\limits_{i=1}^N u_i(t) \quad\laplace\quad U(f) = \sum\limits_{i=1}^N U_i(f)}
\formTab{Kanalraster (bei $\Delta f = const$)}{\Delta f = f_{i+1} - f_i}

\subsubsection{Kohärente Demodulation}
\formTab{Demodulation t-domain}{y(t) = u(t) \cdot \cos(\omega_0 t + \psi) \cdot \beta \formTnQ = u(t) \cdot \cos(\omega_0 t + \psi) \cdot K}
\formTab{Demodulation $\omega$ -/f-domain}{Y(\omega) = \frac{1}{2}X(\omega) \cdot \cos(\psi) \formTn  + \underbrace{\frac{1}{2} X(\omega - 2 \omega_0) \frac{1}{2} \cdot e^{j \psi} + \frac{1}{2} X(\omega + 2 \omega_0) \frac{1}{2} \cdot e^{-j \psi}}_{\text{wird herausgefiltert}}}
\formTab{Typische Normierung}{\alpha_A = \beta = \sqrt{2} \Rightarrow \frac{\alpha_K \beta}{2}=1}
\formTab{Tiefpass Filtereigenschaft}{f_{TP,max} = f_{LP,max} = f_{max,LP} 
\formTnQ f_{max} < f_{max,LP} < 2 f_0 - f_{max}}
\formTab{Orthogonalitätsbedingung}{\psi \neq \frac{\pi}{2} + n \cdot \pi \quad ,\; n \in \Z \quad \text{sonst } v(\omega) = 0} 
\formTab{Identität}{V(\omega) = \frac{1}{2} X(\omega) \cdot \cos(\psi) \quad \formTnQ \laplace \quad v(t) = \frac{1}{2}x(t) \cdot \cos(\psi)}

\subsubsection{Inkohärente Demodulation}
\formTab{Demodulation t-domain}{z(t) = \left[ A_{off} + a(t) \right] \cdot \cos(w_0 t), \; A_{off} + a(t) \overset{!}{>} 0}

\subsection{Quadraturamplitudenmodulation}
Bei der QAM wählt man $x(t) \in \C$ so, dass $X(f)$ asymmetrisch wird und damit die Redundanz der Seitenbänder verschwindet. Es wird ein zusätzlicher Träger der orthogonal zum $\cos$-Träger steht verwendet. Dadurch beeinflussen sich beide Träger nicht.
\formTab{QAM}{u(t) = \alpha_A \left[ x_1(t) \cdot \cos(\omega_0 t) - x_2(t) \cdot \cos(\omega_0 t)\right]} 
Es gilt:
\begin{flalign}
	x(t) = x_{re}(t) + j x_{im}(t) &= x_1(t) + j x_2(t) = x_I (t) + j x_Q(t) + x_N (t) + j x_Q (t) \nonumber \\
	\Rightarrow u(t) &= \Re \lbrace \alpha_A x(t) \cdot e^{j \omega_0 t]} \nonumber \\
	&= \Re \lbrace \alpha_A (x_1 (t) + j x_2 (t)) (\cos(\omega_0 t) + j \cdot \sin(\omega_0 t)) \rbrace \\
	\overset{\alpha_A = 1}{\Rightarrow} u(t) &= x_1(t) \cdot \cos(\omega_0 t) - x_2 (t) \cdot \sin(\omega_0 t)
\end{flalign}
Im Spektrum ergibt sich folglich
\begin{align}
	u(t) &= \Re \left\lbrace \alpha_A x(t) \cdot e^{j \omega_0 t} \right\rbrace \overset{\alpha_A = \sqrt{2}}{=} \frac{\sqrt{2}}{2} \big( x(t) \cdot e^{j \omega_0 t} + \left(x(t) \cdot e^{j \omega_0 t} \right)^* \big) \nonumber\\
	\Rightarrow u(t) \quad &\laplace \quad U(\omega) = \frac{\sqrt{2}}{2} \left( X(\omega - \omega_0) + X^*(-\omega - \omega_0)\right)
\end{align}

\formTab{Demodulation}{v(t) = \beta \cdot u(t) \cdot e^{-j \omega_0 t}}

\subsection{Digitale QAM}
\formTab{Komplexes Symbol}{s_k = s_{k,I} + j s_{k,Q} \quad ,\; s_k \in \frac{1}{\sqrt{2}} \lbrace \pm 1 \pm j \rbrace }
\formTab{Basisbandsignal}{x(t) = \sum\limits_{k=0}^{K-1} s_k g(t-T_s) \quad ,\; T = T_s}
\formTab{Symbolrate}{R_s = \frac{1}{T_s}}
\formTab{Übertragene Bits pro Symbol}{M_b = \ld (M)}
\formTab{Bitrate}{R_b = M_b R_s}

\formTab{Bandbreite (bei Brickstone Impuls)}{B = R_s}
\formTab{Logarithmische Darstellung}{X(f) |_{dB} = 10 \log(X^2(f))}
Das komplexe Basisbandsignal erhält man mit
\begin{align}
	x(t) &= \sum\limits_{k=0}^{K-1} s_k \; g(T-kT_s) = \sum\limits_{k=0}^{K-1} (s_{k,re} + j \; s_{k,im})\;g(t-kT_s) \nonumber\\
	 &= \underbrace{\sum\limits_{k=0}^{K-1} s_{k,re} \; g(t-kT_s)}_{I \in \R} + j\underbrace{\sum\limits_{k=0}^{K-1} s_{k,im}\;  g(t-kT_s)}_{Q \in \R} \quad \in \C
\end{align}

\subsection{Bandbreite für verschiedene $s_k$}
\formTab{Basisbandlage für reelle $s_k$}{B = \frac{1}{2} \; R_s = f_{max}}
\formTab{Basisbandlage für komplexe $s_k$}{B =R_s = 2\cdot f_{max}}
\formTab{Passband für reelle und komplexe $s_k$}{B = R_s =2\cdot f_{max}}

\subsection{Komplexer AWGN Kanal}
\formTab{Kapazität}{C_{AWGN,komplex}(SNR) = 2 \cdot C_{AWGN,reell} (SNR) \formTnQQQ  = \ld (1+SNR)}
\formTab{Rauschleistungsdichte}{N_0 = 2 \sigma^2}
\formTab{Signal-Rausch Abstand}{SNR = \frac{E_s}{N_0}}
\input{./tex/ch2/chapter2.tex}
\input{./tex/ch3/chapter3.tex}
 

\newpage
\chapter{Anhänge}
	\section{Nachwort}
Dieses Dokument versteht sich einzig als Zusammenfassung des HM3 Stoffes auf Basis der Literatur und der Vorlesungsunterlagen aus der HM3 Vorlesung von Prof. Dr. Marcel Griesemer mit einigen zusätzlichen Beispielen. Der Sinn ist einzig mir selbst und meinen Kommilitonen das studieren der Mathematik zu erleichtern. In diesem Sinne erhebe ich keinerlei Anspruch auf das hier dargestellte Wissen, da es sich in großen Teilen nur um Neuformulierungen aus der Literatur, den Vorlesungen und aus dem Begleitkurs vom Mint Kolleg handelt, in dem Frau Dr. Monika Schulz den Stoff aus der HMI und HMII bereits hervorragend zusammengefasst hat. Sollten sich einige Fehler eingeschlichen haben (was sehr wahrscheinlich ist) würde ich mich freuen, wenn man mich per Email (f.leuze@outlook.de) kontaktieren und entsprechende Fehler mitteilen würde.

\nocite{*}
\bibliography{./bib/lit}

\end{document}
